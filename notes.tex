\documentclass{article}[12]


\begin{document}

\section{Introduction}

* Units are things we measure with.  Some units are basic ( you can
measure them directly), others are derived (you get them by combining
other measurements).  You can get ``meters'' that will read out in all
kinds of units, but as much as possible, at least in this class, we
want to stick to the basic, fundamental units.  They are also called
SI units.  I swear to you, that the reason I want you to use them is
because they are \emph{easier} than the American or Imperial units.  

* That said, how do we measure \emph{speed}?  What are it's units?  Is
that a derived unit, or a basic one?

So one of the most common units of measure, that we see every day (if
we drive), is actually not in the SI system, and it's not
fundamental.  What are the fundamental units, of which the derived
unit of speed is composed?

\begin{tabular}{c|c|c|c|c|c}
50 miles & 5280 feet  & 12 inches & 2.54 centimeters & 1 meter & 1 hour \\
\hline
1 hour & 1 mile & 1 foot & 1 inch & 100 cm & 3600 seconds

\end{tabular}


\end{document}