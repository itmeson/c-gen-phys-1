\documentclass{article}[12]


\begin{document}

\section{Introduction}
What is physics?

The big questions of physics are the big questions of the universe:
what is it?  Why is it there?  Did it have to be this way, or could it
have been different?  Physics uses mathematics, experiment, and
explanation to come to terms with things that are almost unimaginable
in scale.  The Large Hadron Collider in Switzerland is smashing
particles called protons into each other at speeds very, very close to
the speed of light, and watching very closely to see all the debris
that come out of the collisions.  They are a little bit different
every time, due, apparently, to some randomness in the ether.

What do they do, when they track these debris?  They trace backward,
finding that the debris they see at the detectors (which, obviously,
can't be quite at the location of the collisions), have moved in
straight lines outward from previous collisions -- that themselves
traced backward into the singular moment of the original collision.
They calculate the momenta of these particles, and make calculations
about what kinds of interactions could have caused those particular
particles, with those particular momenta and directions to have been
produced. And every once in a while, they find that they cannot
explain the traces they see using any of the known particles, or any
of the known ways that those particles can interact with each other.
And if they see one of those strange events enough times (so they are
sure it wasn't a fluke of a mishap with the detectors), and they can
find someone who can suggest an explanation, they propose: ``HEY!!
We've found a BRAND NEW particle!! No one has EVER seen this before.
We are BAD-ASS!!''

The stakes are very high.  Billions of dollars.  The careers of
thousands of scientists and engineers.  But at heart, what they are
doing, is not so different from a traffic engineer -- surveying the
scene of an accident, photographing the final locations of the
vehicles, the specific damage done to the bodies, the tracks on the
pavement, to determine who was speeding or which vehicle crossed the
center-line.  Except that sometimes the answer is: ``A Higgs particle
suddenly came into existence and pushed Car #2 into the path of Car
#1.''

Physics I teaches you the basic principles, techniques, and thought
processes necessary to understand a crash scene or a particle trace
from the LHC.  Along the way, you learn how to analyze building
components for stability, the claims of ``green'' energy purveyors,
and the motions of spinning things.  You'll also practice developing
explanatory models for different phenomena, and testing those models
using experiments you will design yourself.  

Many people dislike science and math because it makes them feel
stupid.  Because scientists are unusually good at saying ``You're
wrong!''  

If you learn nothing else in this class, I want you to learn this:
Being Wrong is Good.  You cannot improve unless you hit the wall of
your current limitations.  Being wrong is a sign that you've hit one
of those walls, and is therefore a signal about where you need to
work.  The goal is of course to become not-wrong, but you cannot get
there without going through some wrongness.  My job is to help you
find those limits without hurting you, so you can push them back.

I believe that this is true of everything worth doing, but it is
particularly clear in math and science, where the whole process works
by proposing models and testing them.  A model cannot be a good model
if there is no way that you could discover it is false.  People have
an unfortunate tendency to seek out evidence that supports their
current belief, missing the evidence that contradicts it.  You are
much more likely to avoid being misled by forming a
prediction/model/explanation/etc. that you think is good, then write
down what kinds of evidence would show that you are WRONG, and go
looking for that.

(Give triple of numbers ``2,4,6'', ask them to consider a rule that
would generate this set, then propose new triplets to help them test
their rule.  I will tell them whether their triplet follows my rule or
not, and they should use this to refine their rule.)



Alright -- so enough of that.  Now I'm going to give you a test!  Yes,
a test.  The point of this test is to help me ensure that I'm actually
teaching you something.  I'm setting up an experiment, and my
hypothesis is that you are going to learn a lot of mechanics this
term.  That's what I want to have happen.  So I've identified a lot of
things that would count as evidence that you have NOT learned any
mechanics, and I will test you on them repeatedly through the term.
Your grade will depend on convincing me, over and over again, that you
have learned specific concepts and skills in mechanics.  The more
things you try to convince me about, the higher your grade can
potentially get.  

But for my tests to work, I have to have some information about where
you're at right now.  Because one possible way that I could be misled
is if you happen to ALREADY know a lot of mechanics.  So this test is
designed to help me find that out.  It's multiple choice, it's hard,
and you should do your absolute best on it -- think about the
questions, they are designed to be tricky.  When you are finished,
bring it up to me, and I have an assignment for you to work on for
next time, that you should get started on.




* Units are things we measure with.  Some units are basic ( you can
measure them directly), others are derived (you get them by combining
other measurements).  You can get ``meters'' that will read out in all
kinds of units, but as much as possible, at least in this class, we
want to stick to the basic, fundamental units.  They are also called
SI units.  I swear to you, that the reason I want you to use them is
because they are \emph{easier} than the American or Imperial units.  

* That said, how do we measure \emph{speed}?  What are it's units?  Is
that a derived unit, or a basic one?

So one of the most common units of measure, that we see every day (if
we drive), is actually not in the SI system, and it's not
fundamental.  What are the fundamental units, of which the derived
unit of speed is composed?

\begin{tabular}{c|c|c|c|c|c}
50 miles & 5280 feet  & 12 inches & 2.54 centimeters & 1 meter & 1 hour \\
\hline
1 hour & 1 mile & 1 foot & 1 inch & 100 cm & 3600 seconds

\end{tabular}


\end{document}
